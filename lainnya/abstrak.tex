\chapter*{ABSTRAK}
\begin{center}
  \large
  \textbf{MINTING \textit{NFT} PADA \textit{SOLANA BLOCKCHAIN} UNTUK TIKET KONSER MUSIK DALAM \textit{MARKETPLACE} MENGGUNAKAN \textit{SMART CONTRACT} BERBASIS \textit{WEB3.0}}
\end{center}
\addcontentsline{toc}{chapter}{ABSTRAK}

% Menyembunyikan nomor halaman
\thispagestyle{empty}

\begin{flushleft}
  \setlength{\tabcolsep}{0pt}
  \bfseries
  \begin{tabular}{ll@{\hspace{6pt}}l}
  Nama Mahasiswa / NRP&:& Rianco Marcellino Andreas / 5024211061\\
  Departemen&:& Teknik Komputer FTEIC - ITS\\
  Dosen Pembimbing&:& 1. Mochamad Hariadi, S.T., M.Sc., Ph.D\\
  & & 2. Dr. Susi Juniastuti, S.T., M.Eng.\\
  \end{tabular}
  \vspace{4ex}
\end{flushleft}

\textbf{Abstrak}

Industri \textit{ticketing} konser musik menghadapi tantangan signifikan terkait pemalsuan tiket, penjualan kembali ilegal, dan kurangnya transparansi. Penelitian ini mengusulkan pengembangan sistem \textit{minting NFT} (\textit{Non-Fungible Token}) berbasis \textit{Solana Blockchain} untuk tiket konser musik, dengan tujuan mengatasi masalah-masalah tersebut. Fokus utama penelitian adalah mengoptimalkan proses \textit{minting NFT} untuk meningkatkan efisiensi, keamanan, dan skalabilitas sistem \textit{ticketing} digital. Metodologi penelitian meliputi desain dan implementasi \textit{smart contract} pada platform \textit{Solana}, pengembangan antarmuka \textit{Web3.0} untuk interaksi pengguna, dan optimasi proses \textit{minting} untuk mengurangi biaya \textit{gas} dan meningkatkan \textit{throughput}. Sistem yang diusulkan diuji melalui simulasi \textit{high-load} dan uji coba pengguna terbatas untuk mengevaluasi performa dan \textit{usability}. Hasil yang diharapkan meliputi peningkatan signifikan dalam keamanan dan autentikasi tiket, manajemen penjualan sekunder yang lebih efisien, dan pengalaman pengguna yang lebih baik dalam pembelian dan penggunaan tiket konser digital. Penelitian ini juga bertujuan untuk memberikan wawasan tentang integrasi teknologi \textit{Solana Blockchain} dan \textit{Web3.0} dalam industri hiburan. Kontribusi utama penelitian ini adalah pengembangan sistem \textit{minting NFT} yang dioptimalkan untuk tiket konser musik, yang berpotensi mentransformasi manajemen dan distribusi tiket digital. Temuan penelitian diharapkan dapat memberikan landasan untuk adopsi lebih luas teknologi \textit{blockchain} dalam industri \textit{ticketing} dan membuka jalan bagi inovasi lebih lanjut dalam pengalaman konser digital.

\vspace{2ex}
\noindent
\textbf{Kata Kunci: \textit{NFT}, \textit{Solana Blockchain}, Tiket Konser Digital, \textit{Smart Contract}, \textit{Web3.0}, Optimasi \textit{Minting}}
